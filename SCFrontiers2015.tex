\documentclass{superfri}

\usepackage{amsfonts}
\usepackage{amssymb}
\usepackage[cmex10]{amsmath}
\usepackage{booktabs}
\usepackage{enumitem}
\usepackage{graphicx}
\usepackage{fancyvrb}
\usepackage{ifthen}
\usepackage{cite}
\usepackage{tabulary}
\usepackage{url}
\usepackage{xspace}
\usepackage[pdfborder={0 0 0}]{hyperref}
\usepackage{verbatim}

\usepackage{color}
\definecolor{yellow}{rgb}{1,1,0}
\definecolor{black}{rgb}{0,0,0}
\definecolor{ltcyan}{rgb}{.75,1,1}
\definecolor{red}{rgb}{1,0,0}
\definecolor{gray}{rgb}{.6,.6,.6}
\definecolor{darkred}{rgb}{0.5,0,0}
\definecolor{darkgreen}{rgb}{0,0.5,0}

% Cite commands I use to abstract away the different ways to reference an
% entry in the bibliography (superscripts, numbers, dates, or author
% abbreviations).  \scite is a short cite that is used immediately after
% when the authors are mentioned.  \lcite is a full citation that is used
% anywhere.  Both should be used right next to the text being cited without
% any spacing.
\newcommand*{\lcite}[1]{~\cite{#1}}
\newcommand*{\scite}[1]{~\cite{#1}}

\newcommand{\etal}{et al.\xspace}

\newcommand*{\keyterm}[1]{\emph{#1}}

\newcommand{\fix}[1]{{\color{red}\textsc{[#1]}}}
%\newcommand{\fix}[1]{}

% Avoid putting figures on their own page.
\renewcommand{\textfraction}{0.05}
\renewcommand{\topfraction}{0.95}
\renewcommand{\bottomfraction}{0.95}

% Make sure this is big enough so that only big figures end up on their own
% page but small enough so that if a figure does have to be on its own
% page, it won't push everything to the bottom because it's not big enough
% to have its own page.
\renewcommand{\floatpagefraction}{.75}

\begin{document}

%\classify{MSC?}
\author{\fix{I.M.~Scientist}\footnote{\label{susu}South Ural State University}, \fix{U.R.~Author}\footnoteref{susu}}

\title{\fix{Hank's talk title.}}

\maketitle{}

\begin{abstract}%
  \fix{Start with Hank's abstract, which I do not have right now.}

  \keywords{scientific visualization, exascale}
\end{abstract}


\section*{Introduction}
\label{sec:Introduction}

\fix{slide of doom}

\fix{push to many-core}

\fix{current Vis SW inadequate}

\fix{how to get portable performance}

\section{Data Parallel Primitives}

\fix{what they are, how they help}

\fix{discuss Blelloch briefly}

\fix{DAX/EAVL/PISTON, leading into VTK-m}


\section{Patterns for Data Parallel Visualization}

\fix{data parallel primitives in action (focusing on ``re-thinking algs'')}

\fix{show an example of isosurface}


\section{Results}

\fix{Show ray-tracing results}


\section{Conclusion}

\fix{conclude DPP is viable}


\ack{
  This material is based upon work supported by the U.S. Department of
  Energy, Office of Science, Office of Advanced Scientific Computing
  Research, under Award Numbers 10-014707, 12-015215, and 14-017566.

  Sandia National Laboratories is a multi-program laboratory managed and
  operated by Sandia Corporation, a wholly owned subsidiary of Lockheed
  Martin Corporation, for the U.S. Department of Energy's National Nuclear
  Security Administration under contract DE-AC04-94AL85000.
}


\bibliographystyle{plain}
%\bibliography{SCFrontiers2015}

\end{document}
